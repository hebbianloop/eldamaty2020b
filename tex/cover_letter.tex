\documentclass{letter}
\usepackage[margin=0.75in]{geometry}
%\usepackage{fancyhdr}
%\pagestyle{fancy}
%\rhead{\thepage}
%\cfoot{}
\signature{Shady El Damaty}
%\address{Center for Functional \& Molecular Imaging \\ 3900 Reservoir Road \\ Preclinical Sciences Building \\ Suite LM14 \\ Washington, D.C. \\ U.S.A.}
\begin{document}
\begin{letter}{}
\opening{Dear Editor(s) of Developmental Psychobiology:}
I am pleased to submit to you my first, first authored paper entitled, ``Introducing an Adolescent Cognitive Maturity Index and Tracking Vulnerability in Emerging Adulthood.'' 
\\ \\
To the authors' best knowledge, this manuscript details the first systematic approach for estimating ``cognitive age'' in developing adolescents. We apply well-known open-source statistical methods to address a novel problem with implications for significant social impact in real-world settings. Our work demonstrates the importance of neuropsychological task performance for building rigorous models that bridge the gap between behavioral and cognitive neuroscience,\footnote[1]{Niv, Y. (2020, October 22). The primacy of behavioral research for understanding the brain. https://doi.org/10.31234/osf.io/y8mxe} specifically to tackle the difficult and important challenge of tracking maturation from adolescence into adulthood. 

Cognitive skills training is known to be an effective instrument of youth violence prevention,\footnote[2]{Lipsey, M. W., Howell, J. C., Kelly, M. R., Chapman, G., \& Carver, D. (2010). Improving the effectiveness of juvenile justice programs. Washington DC: Center for Juvenile Justice Reform at Georgetown University.} however direct connections between prevention and current working models of adolescent neurocognitive development have remained superficial, despite a decade of research.\footnote[3]{Johnson, S. B., Blum, R. W., \& Giedd, J. N. (2009). Adolescent maturity and the brain: the promise and pitfalls of neuroscience research in adolescent health policy. Journal of Adolescent Health, 45(3), 216-221.} Our work provides a detailed quantitative elaboration of age-related change in inhibitory control, risk taking, and emotional processing skills and how the maturation of these skills relates to risk factors in the difficult transition from adolescence into emerging adulthood. We believe the described methodology for estimating a quantitative metric of the elusive concept of maturity will have a substantial impact on the field of adolescent cognitive development, as well as informing translational research programs in violence and adverse outcome prevention. 

We would like to disclose that an early version of the manuscript has been submitted as a pre-print on PsyArXiv\footnote[4]{El Damaty, S., Darcey, V., McQuaid, G., Stoianova, M., Fesalbon, M., Mucciarone, V., … VanMeter, J. (2020, October 26). Introducing an Adolescent Cognitive Maturity Index and Tracking Vulnerability in Emerging Adulthood. https://doi.org/10.31234/osf.io/6uwrp} but has not been submitted or published at any journal. All data collection and analysis were performed in compliance with regulatory guidelines and requirements for human subjects research.

I hope you enjoy the contents and insights in this paper as much as we have in producing this scholarly product for your enlightened consumption. Thank you and the reviewers for considering this paper for publication in your journal,

\closing{On behalf of the Adolescent Development Study Team,}

\end{letter}
\end{document}